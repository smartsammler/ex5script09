% 2009-10-29
\nr

\begin{enumerate}
\item sc $\rightarrow$ ,,simple cubic{}`` $\rightarrow$ kubisch primitive
Gitter \\
$CsCl$ \\
$R=n_{1}\vec{a}+n_{2}\vec{b}+n_{3}\vec{c}$\\
Basis $\searrow$\\
2 Atome $\left(0,\,0,\,0\right)$, $\left(\dfrac{1}{2},\,\dfrac{1}{2},\,\dfrac{1}{2}\right)$
\\
$\left.r_{i}\right|_{Cs}=1,65\,\AA$ \\
$\left.r_{i}\right|_{Cl}=1,81\,\AA$ \\
$p.\, V.=\text{Packungsverhältnis }$\\
$\left.p.\, V.\right|_{sc}\approx0,52$
\item bcc $\rightarrow$ ,,body centrered cubic{}`` $\rightarrow$ kubisch
raumzentriertes Gitter\\
Metalle: Na, Fe, Cr \\
Raumfüllung\\
$\left.p.\, V.\right|_{bcc}=0,68$ \\
$\sim30\,\%$ aller Elemente \\
Basis $\rightarrow$ 1 Atom \\
Gitterpunkte: $\left(0,\,0,\,0\right)$, $\left(\dfrac{1}{2},\,\dfrac{1}{2},\,\dfrac{1}{2}\right)$\\

\item fcc $\rightarrow$ ,,face centered cubic{}`` $\rightarrow$ kubisch
flächenzentrierte Gitter\\
$\sim30\,\%$ aller Elemente\\
z.\,B.: Cu, Ag, Au, Ni\\
4 Gitterpunkte: $\left(0,\,0,\,0\right)$, $\left(\dfrac{1}{2},\,0,\,\dfrac{1}{2}\right)$,
$\left(\dfrac{1}{2},\,\dfrac{1}{2}\,,0\right)$, $\left(0,\,\dfrac{1}{2},\,\dfrac{1}{2}\right)$\\
$\left.p.\, V.\right|_{fcc}=0,74$ \\
(Abb. Natriumchlorid- Struktur fcc) \end{enumerate}
\begin{description}
\item [{$NaCl$:}] 2 Atome Basis:\\
$\begin{array}{c}
\left(0,\,0,\,0\right)\leftarrow\\
\left(\dfrac{1}{2},\,\dfrac{1}{2},\,\dfrac{1}{2}\right)\leftarrow\end{array}\text{Untergitter }$ 
\item [{$Diamant$:}] 2 Basisatome\\
$\left(0,\,0,\,0\right)$; $\left(\dfrac{1}{4},\,\dfrac{1}{4},\,\dfrac{1}{4}\right)$
\item [{$ZnS$:}] Zinkblende \\
$\underset{Zn}{\left(0,\,0,\,0\right)}$; $\underset{S}{\left(\dfrac{1}{4},\,\dfrac{1}{4},\,\dfrac{1}{4}\right)}$
\\
(Abb.: Diament- und Zinkblende-Struktur fcc)
\item [{\setcounter{enumi}{3} }]~\end{description}
\begin{enumerate}
\item hcp $\rightarrow$ ,,hexagonal close packed{}`` $\rightarrow$ hexagonal
dichteste Kugelpackung\\
$\sim35\,\%$ aller Elemente \\
$\left.p.\, V.\right|_{hcp}=0,74$ \\
(Abb.: Kugelpackung: hexagonal oder kubisch? hcp und fcc)\\
(Abb.: Primitive Elementarzelle)
\end{enumerate}

\paragraph{Wigner- Seitz- Zelle}
\begin{itemize}
\item Elementarzelle mit Gitterpunkt im Zentrum der Elementarzelle
\item lückenlose Bedeckung
\end{itemize}
Mittelsenkrechte 


\paragraph{Millersche Idizes}

1839 Willhelm Miller 

3 ganzzahlige Indizes

$\left(h,\, k,\, l\right)$

$\dfrac{1}{h}\vec{a}_{1},\,\dfrac{1}{k}\vec{a}_{2},\,\dfrac{1}{l}\vec{a}_{3}$ 

Basisvektoren schneiden die Ebenen an den Kehrwerten $\dfrac{1}{h},\,\dfrac{1}{k},\,\dfrac{1}{l}$ 

$\left\{ h.\,.\, l\right\} $ Beschreibung der Ebene 

Negative Indizes z.\,B.: $\left(\bar{1},\,0,\,\bar{1}\right)$

% \includegraphics[scale=0.35,bb = 0 0 200 100, draft, type=eps]{images/img36.gif}

%
\begin{figure}


\caption{}



\end{figure}
% \includegraphics[scale=0.5,bb = 0 0 200 100, draft, type=eps]{images/image001.gif}


\paragraph{Reziprokes Gitter}

$\overset{Gitter}{\vec{A}=m_{1}\vec{a}_{1}+m_{2}\vec{a}_{2}+m_{3}\vec{a}_{3}\Rightarrow\overset{\text{reziprokes Gitter}}{\vec{B}=n_{1}\vec{b}_{1}+n_{2}\vec{b}_{2}}}+n_{3}\vec{b}_{3}$

$\vec{b}_{1},\,\vec{b}_{2},\,\vec{b}_{3}$ als Basisvektoren 

$\vec{b}_{1}=2\pi\dfrac{\vec{a}_{2}\times\vec{a}_{3}}{\underbrace{\vec{a}_{1}\cdot\left(\vec{a}_{2}\times\vec{a}_{3}\right)}_{V_{a}\text{ Volumen}}}$

$\vec{b}_{2}=2\pi\dfrac{\vec{a}_{3}\times\vec{a}_{1}}{V_{a}}$

$\vec{b}_{3}=2\pi\dfrac{\vec{a}_{1}\times\vec{a}_{2}}{V_{a}}$ 

$V_{B}=\vec{b}_{1}\cdot\left(\vec{b}_{2}\times\vec{b}_{3}\right)=\dfrac{\left(2\pi\right)^{3}}{V_{a}}$

für Rechtwinkliges Kristallsystem

$\vec{b}_{1,2,3}=\dfrac{2\pi}{a_{1,2,3}^{2}}\vec{a}$

Eigenschaft

$\vec{a}_{i}\cdot\vec{b}_{i}=2\pi\delta_{ij}$

$\vec{A}\cdot\vec{B}=2\pi n$ 

$\begin{array}{cc}
sc\rightarrow & sc\\
bcc\rightarrow & fcc\\
fcc\rightarrow & bcc\end{array}$